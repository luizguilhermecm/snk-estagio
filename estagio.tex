% vim: set ai tw=80 fileencoding=utf8: 
%-------------------------------------------------------------------------------
\documentclass[12pt]{article}

\usepackage{sbc-template}

\usepackage[Algoritmo,plain]{algorithm}
\usepackage{algorithmic}
\usepackage{listings}
\usepackage{color}
\usepackage{morefloats}
\definecolor{red}{rgb}{0.6,0,0} % for keywords
\definecolor{green}{rgb}{0.25,0.5,0.35} % comments
\definecolor{purple}{rgb}{0.5,0,0.35} % strings
\definecolor{blue}{rgb}{0.25,0.35,0.75} % doc

\lstset{language=c,
basicstyle=\ttfamily\small,
keywordstyle=\color{red}\bfseries,
stringstyle=\color{purple},
commentstyle=\color{green},
morecomment=[s][\color{blue}]{/**}{*/},
numbers=left,
numberstyle=\tiny\color{black},
stepnumber=1,
numbersep=7pt,
tabsize=2,
showspaces=false,
showstringspaces=false}



\usepackage[T1]{fontenc}		% Selecao de codigos de fonte.
\usepackage[utf8]{inputenc}		% Codificacao do documento
%\usepackage[brazil]{babel} % Tradução de seções, bibliografia, etc.
%\usepackage[latin1]{inputenc} % Caracteres ISO-8859-1
\usepackage{graphicx} % Inserção de imagens
\usepackage{amssymb} % Símbolos do tipo \mathbb
\usepackage{amsfonts} % Símbolos do tipo \mathfrak
\usepackage{amsmath} % Comandos do tipo \dfrac
\usepackage{booktabs} % Pacote de formatação de tabelas
\usepackage{multirow} % Permite multirow em tabelas ==> faz merge de linhas em células da tabela
\usepackage{url} % Formata urls como hiperlinks
\usepackage{hyperref} % Cria hiperlinks
\hypersetup{colorlinks=true,linkcolor=black,citecolor=black,urlcolor=black} % Formata a aparência dos hiperlinks
\usepackage{pdfpages} % Permite incluir páginas PDF geradas por outros aplicativos
     
\renewcommand{\tablename}{Tabela}

\sloppy %para evitar palavras longas fora das margens

\title{Otimização de código utilizando técnicas de transformação de loops}
\author{Luiz Guilherme Castilho Martins\inst{1}, Wesley Attrot\inst{1} }

\address{Departamento de Computação -- Universidade Estadual de Londrina (UEL)\\
  Caixa Postal 10.011 -- CEP 86057-970 -- Londrina -- PR -- Brasil
  \email{luizgui@gmail.com, wesley@uel.br}
}
\begin{document} 
\maketitle
\begin{abstract}
        The goal of this work is the optimization of computation of loops,
        computationally expensive programs in general spend most of their time
        in execution of loops.
        Loops with a big number of iterations over a big data arrays, those are
        the ones that spend most time in execution and has a considerable
        potential of results after optimization.
        Tecniques of loop transformation as, loop unswitching, loop fission,
        loop fusion, loop unrolling, among others, was used to optimization.
        Analysis of data dependence was necessary to keep the meaning of the 
        program after the optimizations.
\end{abstract}

\begin{resumo} 
        O objetivo deste trabalho é otimizar a computação de \textit{loops}, os 
        quais são conhecidos por serem responsáveis pelo maior gasto 
        computacional de um programa. 
        \textit{Loops} com grande número de iterações sobre grandes vetores de 
        dados são os que apresentam grande gasto computacional e 
        consequentemente maior chance de resultado na otimização.
        Técnicas de transformações de \textit{loops} como, \textit{loop 
        unswitching, loop fission, loop fusion, loop unrolling}, entre outras, 
        foram utilizadas para otimização.
        A análise de dependência de dados foi utilizada para manter o 
        significado do programa depois das otimizações.
\end{resumo}
%===============================================================================
\section{Introdução}



Diante da  dificuldade de se melhorar processadores \textit{single-core} devido 
ao alto consumo de energia e as altas
temperaturas, a indústria de processadores adotou como solução à esses problemas, 
o desenvolvimento de processadores \textit{multi-core} \cite{Suleman:2008}, 
além disso também considerou-se o poder de processamento quando se utilizam 
vários processadores \textit{single-core} 
trabalhando simultaneamente, os quais oferecem grande desempenho, maior 
eficiênc ia energética e menor custo \cite{realcloudWehner}.
Porém as aplicações desenvolvidas para a arquitetura \textit{single-core} não 
utilizam-se do poder computacional dos processadores \textit{multi-core}.
Para que as aplicações possam usufruir do alto desempenho que estes 
processadores oferecem, as aplicações devem ser 
divididas em múltiplas partes para que possam ser executadas em paralelo.

A paralelização de código permite ao programador resolver problemas com maior 
eficiência, porém projetar e codificar um programa paralelo continua sendo uma 
tarefa difícil, uma vez que além de dividi-lo em pequenas tarefas, deve-se 
considerar a concorrência entre elas, as dependências de dados, entre outros 
fatores que dificultam o trabalho de 
paralelização \cite{aipp}.

Programas computacionalmente custosos em geral gastam a maior parte do tempo na
computação de \textit{loops} extrair paralelismo destes \textit{loops} pode 
ser significativo no desempenho de um programa.
Diante deste desafio uma das soluções é a utilização de transformações de
\textit{loops} para melhorar seu desempenho.
\cite{hpcfpc}.

O restante desse documento está organizado da seguinte forma: 
A seção 2 apresentam os objetivos do trabalho.
A seção 3 apresenta a fundamentação teórico-metodológica.
A seção 4 apresentam os experimentos e os resultados obtidos. 
A seção 5 apresenta a conclusão deste trabalho.
%===============================================================================
\section{Objetivo}

O objetivo principal deste trabalho foi obter ganho de desempenho na execução 
de programa através da aplicação de técnicas de transformação de \textit{loops}.

Além disso, uma análise de dependência de dados foi necessária, uma vez que a 
utilização das técnicas exige que o programa seja alterado e
com isso as dependências de dados do programa podem ser violadas, alterando o 
significado do programa.

%===============================================================================
\section{Fundamentação Teórico-Metodológica}
\label{sec:fundamentacao}

%===============================================================================

\subsection{Taxonomia de Flynn}

Taxonomia de Flynn, proposta em 1966 \cite{Flynn:1966} por Michael 
Flynn e expandida em 1972 \cite{Flynn:1972}, é uma das formas de classificar o 
paralelismo disponível no processador.  

Taxonomia de Flynn utiliza o conceito de sequência de objetos ou ações, que são
chamados de \textit{stream}.
Flynn introduziu dois tipos de \textit{stream}: o 
\textit{stream} de instrução e também o \textit{stream} de dados. 

O \textit{stream} de instrução consiste em uma sequência de instruções. 
Uma instrução ou \textit{instruction word (IW)} é uma cadeia de 0's e 1's que 
representa a menor operação visível ao programador e que será executada pelo 
processador. 
Uma instrução pode conter uma ou mais operações e devido a esta peculiaridade,
alguns autores utilizam \textit{instruction} para instruções que contenham 
apenas uma operação e \textit{instruction word} para instruções que contenham 
mais de uma operação.

Existem no entanto quatro combinações de \textit{streams} que descrevem as 
arquiteturas de computadores mais comuns \cite{Flynn:1996}:

\begin{enumerate}
        \item \textbf{SISD:} \textit{Single Instruction, Single Data};
        \item \textbf{SIMD:} \textit{Single Instruction, Multiple Data};
        \item \textbf{MISD:} \textit{Multiple Instruction, Single Data};
        \item \textbf{MIMD:} \textit{Multiple Instruction, Multiple Data}.
\end{enumerate}

Cada combinação de \textit{streams} caracteriza uma classe de arquitetura 
e cada uma destas classes possui seus tipos de paralelismo específicos \cite{eopc}.

%===============================================================================
\subsection{Dependência de dados}

A análise da dependência de dados em paralelização de código é crucial para que 
durante a execução de um algoritmo não sejam realizadas operações utilizando-se 
de dados incorretos devido à escritas ou leituras indevidas. 
Entre as técnicas utilizadas para análise e controle da dependência de dados 
está o grafo de dependência, garantindo a produção e consumo de dados de maneira 
correta \cite{aapc}.
Segundo \cite{ocfma} são três os tipos dependência de dados classificados 
como \textit{Load-Store}:
\begin{itemize}
        \item \textit{True dependence}: dependência do tipo leitura depois da 
                escrita ou \textit{Read After Write (RAW)};
        \item \textit{Antidependence}: dependência do tipo escrita depois da 
                leitura ou \textit{Write After Read (WAR)};
        \item \textit{Output dependence}: dependência do tipo escrita depois da 
                escrita ou \textit{Write After Write (WAW)}.
\end{itemize}
Transformações podem ser feitas onde existem dependências de dados, uma vez que 
o programa resultante da transformação seja equivalente ao programa inicial. 
A definição de equivalência diz que dois programas são equivalentes se com a 
mesma entrada ambos terão a mesma saída tendo sido executados na mesma 
ordem \cite{ocfma}.

%===============================================================================
\subsection{Reestruturação de loops}

Os \textit{loops} são responsáveis por grande custo computacional, assim, a 
reestruturação de \textit{loops} é uma técnica muito importante em paralelização 
de código tanto quanto é uma tarefa difícil \cite{Huang2000259}.

Serão intrutuzidas a seguir as principais técnicas de transformação de 
\textit{loops}.

%===============================================================================
\subsubsection{Reordenação de Declarações}

A reordenação de declarações pode ser utilizada para amortizar a latência da 
memória, melhorar o \textit{data locality} movendo declarações ou \textit{loops}
que utilizem os mesmos dados próximas uma das outras \cite{McKinley:1996}
A reodenação é considerada válida sempre que respeitar a dependência de dados e 
não modificando o significado do programa.

O uso de reordenação de declarações é uma técnica útil uma vez que através dela 
é possivel a aplicação de outras técnicas.

%===============================================================================
\subsubsection{Loop Unswitching}

\textit{Loop unswitching} é uma técnica simples e eficiênte cujo o objetivo 
é remover uma ou mais declarações de comparação que esteja dentro do
corpo do \textit{loop}.
Remover uma condição de um \textit{loop} faz com que a condição envolva um, dois
ou mais \textit{loops}, ou seja, para cada resultado da condição removida.
A condição deve obrigatoriamente ser independente do \textit{loop}, ou seja, o
resultado da condição é sempre o mesmo não dependendo das iterações do 
\textit{loop} \cite{Ghodrat:2008}.

A vantagem de \textit{loop unswitching} é a redução da frequência da execução
da condição, sendo executada apenas uma vez.
A sua desvantagem de é o aumento da estrutura do
\textit{loop} prejudicando a legibilidade do código.

%===============================================================================
\subsubsection{Loop Peeling}

\textit{Loop peeling} remove a primeira ou a última iteração de um
\textit{loop}, colocando-a para fora do corpo \textit{loop}, fazendo com que um
\textit{loop} que iniciaria as iterações a partir de $0$ inicie a partir de $1$.
Essa técnica pode ser utilizada para ajudar o \textit{index} do \textit{loop}
para a aplicação de \textit{loop fusion} \cite{Song:2004}.

%===============================================================================
\subsubsection{Index Set Splitting}

\textit{Index set splitting} ou \textit{loop peeling} é uma generalização de 
\textit{loop peeling}. 
Sua aplicação se resume a dividir o \textit{index} do \textit{loop} em duas
partes e então duplicar o \textit{loop}, sendo que um irá executar a primeira
parte das iterações e o segundo irá até a última iteração.
Está técnica é útil para remover declarações de comparação do corpo do
\textit{loop} que utilizem o \textit{index} do \textit{loop}
\cite{Tasharofi:2010}.

%===============================================================================
\subsubsection{Loop Unrolling}

\textit{Loop unrolling} baseia-se em replicar o corpo do \textit{loop} fazendo
com que o número de iteração seja reduzido, diminuindo o \textit{overhead} de
verificação da condição de saída do \textit{loop}.
A generalização desta técnica pode ser aplicada duplicando-se o corpo do
\textit{loop} $n$ vezes e assim dividindo a quantidade de iterações por $n$.
Em casos onde o número de iterações não for conhecido, o uso desta técnica
implicará em tratamentos extras para garantir que todas iterações foram
computadas, em casos de uso da generalização de \textit{loop unrolling} com $n$
grande suficiente é necessário um segundo \textit{loop} para terminar 
as iterações.
O uso de \textit{loop unrolling} é sempre válida se mantiver a
dependência e o significado do programa \cite{Dragomir:2009}.

%===============================================================================
\subsubsection{Loop Fusion}

\textit{Loop fusion} transforma dois \textit{loops} adjacentes com o mesmo
espaço de iteração em um \textit{loop} simples.
Esta técnica aplicada a dois \textit{loops} que referenciam o mesmo conjunto de
dados apresenta uma melhora no \textit{temporal locality}\cite{Wolf:1996}.

O uso de \textit{loop fusion} em dois ou mais \textit{loops} requer que sejam 
adjacentes, com limites de iteração compatíveis e sem relação de dependência 
entre eles. 
Esta seria uma aplicação trivial de \textit{loop fusion}.
Em casos onde algumas das condições não estejam satisfeitas será necessário a o
uso de outras técnicas para possibilitar o uso de \textit{loop fusion}.

\textit{Loops} com diferentes limites de iteração, poderá ser utilizado
\textit{loop peeling, loop splitting} ou até \textit{loop unrolling} para que
fiquem compatíveis. 
O uso destas podem trazer desvantagens como a adição de código extra e
aumentar a complexidade do \textit{loop}, fazendo com que a legibilidade do
código seja comprometida.
Em casos onde adicionar uma declaração de condição faz-se necessário, deverá ser
considerado o custo adicional sobre o teste condicional.

Se os \textit{loops} não forem adjacentes, será necessário realizar uma
reorndeção de declarações.

O uso de \textit{loop fusion} é sempre válido se todas as relações de 
dependência entre os \textit{loops} forem preservadas e não alterar o significado do
programa. 

%===============================================================================
\subsubsection{Loop Fission}

\textit{Loop fisson} transforma um \textit{loop} simples e o divide em dois
\textit{loops} menores. 
Uma vez que um \textit{loop} é divido, o \textit{index} do \textit{loop} ser á
utilizado em ambos os \textit{loop}.
O uso de \textit{loop fission} pode melhorar o \textit{memory locality}, uma vez
que a quantidade de dados acessados por iteração é menor \cite{hpcfpc}.


%===============================================================================
\section{Experimentos e Resultados}
\label{sec:experimentos}

Para a realização dos experimento foi escolhido o programa 
\textit{wat\footnote{https://github.com/luizguilhermecm/wat}} 
um equalizador de áudio de 1 e 2 canais.
A escolha de \textit{wat} foi motivada pela fato deste possuir
\textit{loops} com grande espaço de iteração sobre grande quantidade 
de dados, estas são as características de \textit{loops} custosos, logo, estes 
também possuem potencial para obter ganho de desempenho através da aplicação de 
tranformações de \textit{loops}.

Os experimentos realizados foram executados em uma máquina com as seguintes
configuração: processador, \textit{Intel Core 2 Duo, 2.4 GHz}, 
memória \textit{4 GB 1064 MHz DD3} e 
\textit{HD 256 GB Solid State SATA Drive}.

Para compilação dos resultados, foram realizados dez medições de tempo e então
feita a média geométrica destes tempos.

\subsection{Experimentos}

Todos os \textit{loops} de \textit{wat} que apresentam as características 
de \textit{loops} custosos foram utilizados nos experimentos. 

\begin{algorithm}[H]
  \caption{\textit{Loop} extraído de \textit{wat}.}
    \label{read_data_orig}
\lstinputlisting[language=c]{src/read_data_orig.c}
\end{algorithm}


Dos seis \textit{loops} analisados, os
algoritmos~\ref{read_data_orig},~\ref{fix_data_orig},~\ref{convert_double_orig}
e~\ref{back_data_orig}, tinham declaração de condição no corpo
do \textit{loop}.
O fato das comparações estarem dentro do corpo do \textit{loop} implica que
estas comparações são realizadas a cada iteração do \textit{loop}.
Uma vez que estas verificações não depende da iteração do \textit{loop} e também
não dependem dos dados calculados no corpo do \textit{loop}, logo, estas
declarações de comparação são independente do \textit{loop} e poderiam ser
removidas.

\begin{algorithm}[H]
  \caption{\textit{Loop} extraído de \textit{wat}.}
    \label{fix_data_orig}
\lstinputlisting[language=c]{src/fix_data_orig.c}
\end{algorithm}


A técnica \textit{loop unswitching} foi aplicada em todos os casos, 
assim removendo as declarações do corpo de cada \textit{loop}. 

Os resultados obtidos em todas as quatro aplicações de \textit{loop unswitching} 
foram positivos em questão de desempenho. 
Somente o uso de \textit{loop unswitching} nestes \textit{loops}, significou uma
melhora 5\% na computação de cada \textit{loop}.

A técnica \textit{loop unrolling} foi aplicada em todos os algoritmos na
tentativa de diminuir o número de iterações dos \textit{loops} e também melhorar
o \textit{memory locality}. 
Os resultados obtidos por \textit{loop unrolling} foram expressivos em casos
onde o \textit{loop} iterava sobre uma diversidade pequena de dados.
Nos casos onde a quantidade de dados diferentes eram grandes
o desempenho de \textit{loop unrolling} foi pouco ou nada significativo.

\begin{algorithm}[H]
  \caption{\textit{Loop} extraído de \textit{wat}.}
    \label{convert_double_orig}
\lstinputlisting[language=c]{src/convert_double_orig.c}
\end{algorithm}

\begin{algorithm}[H]
  \caption{\textit{Loop} extraído de \textit{wat}.}
    \label{save_file_orig}
\lstinputlisting[language=c]{src/save_file_orig.c}
\end{algorithm}


O uso mais expressivo de \textit{loop unrolling} foi no \textit{loop} do
algoritmo~\ref{save_file_orig} que é o mais
custoso do programa, devido ao gargalo de escrita em disco.
Foi aplicado \textit{loop unrolling} replicando o corpo do \textit{loop} 2, 4, 8
e 16 vezes. O melhor desempenho foi ao replicar 4 vezes, apresentando um ganho
de desempenho de quase 10\%.

O uso da técnica \textit{loop fission} mostrou-se bastante desafiadora, uma vez
que apresentou altos e baixos no desempenho.
O uso de \textit{loop fission} sem utilizar outras técnicas em conjunto não
apresentaram melhoras, embora o seu uso combinado com \textit{loop unrolling}
potencializa as características de ambas as técnicas, por acessar um volume
maior do mesmo dado na memória.

\textit{Loop fission} foi aplicada nos
algoritmos~\ref{read_data_orig},~\ref{back_data_orig} e~\ref{fix_data_orig}
apresentando melhora de desempenho dos algoritmos~\ref{back_data_orig}
e~\ref{fix_data_orig}.

\begin{algorithm}[H]
  \caption{\textit{Loop} extraído de \textit{wat}.}
    \label{back_data_orig}
\lstinputlisting[language=c]{src/back_data_orig.c}
\end{algorithm}


\textit{Index set splitting} foi responsável pelo maior ganho individual de
desempenho, uma vez que o algoritmo~\ref{equalize_orig} tem um \textit{loop} 
com onze condições dependentes do \textit{loop} aninhadas. 
A aplicação de \textit{index set splitting} no algoritmo~\ref{equalize_orig} 
dividiu-o em onze \textit{loops} sem que nenhum deles tivessem declarações 
condicionais no corpo do \textit{loop}. 
Isso significou um ganho de desempenho de 50\% no custo da computação daquele
\textit{loop}.

\textit{Loop fusion} aplicada no algoritmo~\ref{convert_double_orig} não
apresentou resultado positivo, uma vez que tornou o corpo do \textit{loop}
grande demais e fazendo com que ajustes custosos fossem necessários para ajustar
as iterações dos \textit{loops}.

As
tabelas~\ref{tabela_read_data},~\ref{tabela_fix_data},~\ref{tabela_equalize},~\ref{tabela_back_data},~\ref{tabela_convert_double}
e~\ref{tabela_save_file} apresentam os resultados da aplicação das técnicas em
cada um dos \textit{loops} retirados de \textit{wat}.

\begin{algorithm}[H]
  \caption{\textit{Loop} extraído de \textit{wat}.}
    \label{equalize_orig}
\lstinputlisting[language=c]{src/equalize_orig.c}
\end{algorithm}

\begin{table}[H]
  \caption{Resultados das técnicas sobre o algoritmo~\ref{read_data_orig}.}
  \label{tabela_read_data}
\begin{center}
  \begin{tabular}{|c|c|}
    \hline
    Técnicas aplicadas & Tempo de execução\\
    \hline
    nenhuma & 470 \textit{ms} \\
    \hline
    \textit{loop unswitching} & 448 \textit{ms} \\
    \hline
    \textit{loop unswitching, loop unroll} & 444 \textit{ms} \\
    \hline
    \textit{loop unswitching, loop fission} & 501 \textit{ms} \\
    \hline
    \textit{loop unswitching, loop fisson, loop unroll} &  480 \textit{ms} \\
    \hline
  \end{tabular}
\end{center}
\end{table}


\begin{table}[H]
  \caption{Resultados das técnicas sobre o algoritmo~\ref{fix_data_orig}.}
  \label{tabela_fix_data}
\begin{center}
  \begin{tabular}{|c|c|}
    \hline
    Técnicas aplicadas & Tempo de execução\\
    \hline
    nenhuma & 485 \textit{ms} \\
    \hline
    \textit{loop unswitching} & 462 \textit{ms} \\
    \hline
    \textit{loop unswitching, loop unroll} & 443 \textit{ms} \\
    \hline
    \textit{loop unswitching, loop fission} & 485 \textit{ms} \\
    \hline
    \textit{loop unswitching, loop fisson, loop unroll} & 435 \textit{ms} \\
    \hline
  \end{tabular}
\end{center}
\end{table}

\begin{table}[H]
  \caption{Resultados das técnicas sobre o algoritmo~\ref{equalize_orig}.}
  \label{tabela_equalize}
\begin{center}
  \begin{tabular}{|c|c|}
    \hline
    Técnicas aplicadas & Tempo de execução\\
    \hline
    nenhuma & 240 \textit{ms} \\
    \hline
    \textit{index set splitting} & 145 \textit{ms} \\
    \hline
    \textit{index set splitting, loop unroll} & 75 \textit{ms} \\
    \hline
  \end{tabular}
\end{center}
\end{table}

\begin{table}[H]
  \caption{Resultados das técnicas sobre o algoritmo~\ref{back_data_orig}.}
  \label{tabela_back_data}
\begin{center}
  \begin{tabular}{|c|c|}
    \hline
    Técnicas aplicadas & Tempo de execução\\
    \hline
    nenhuma & 150 \textit{ms} \\
    \hline
     \textit{loop unswitching} & 147 \textit{ms} \\
    \hline
     \textit{loop unswitching, loop unroll} & 148 \textit{ms} \\
    \hline
     \textit{loop unswitching, loop fission} & 150 \textit{ms} \\
    \hline
    \textit{loop unswitching, loop fisson, loop unroll} & 144 \textit{ms} \\
    \hline
  \end{tabular}
\end{center}
\end{table}

\begin{table}[H]
  \caption{Resultados das técnicas sobre o algoritmo~\ref{convert_double_orig}.}
  \label{tabela_convert_double}
\begin{center}
  \begin{tabular}{|c|c|}
    \hline
 Técnicas aplicadas & Tempo de execução\\
    \hline
    nenhuma & 494 \textit{ms} \\
    \hline
     \textit{loop unswitching} & 450 \textit{ms} \\
    \hline
    \textit{loop unswitching, loop unrolling}& 422 \textit{ms} \\
    \hline
    \textit{loop unswitching, loop fusion} & 477 \textit{ms} \\
    \hline
  \end{tabular}
\end{center}
\end{table}

\begin{table}[H]
  \caption{Resultados das técnicas sobre o algoritmo~\ref{save_file_orig}.}
  \label{tabela_save_file}
\begin{center}
  \begin{tabular}{|c|c|}
    \hline
    Técnicas aplicadas & Tempo de execução\\
    \hline
    nenhuma & 6910 \textit{ms} \\
    \hline
    \textit{loop unrolling} & 6296 \textit{ms} \\
    \hline
  \end{tabular}
\end{center}
\end{table}



\subsection{Resultados Experimentais}

Considerando os resultados experimentais da tabela~\ref{tabela_resultados}.
O tempo total de execução dos \textit{loops} analisados foi de 8750 \textit{ms},
após a aplicação das técnicas de tranformação de \textit{loops} o tempo total
foi de 7816 \textit{ms}, ou seja, houve uma melhora de 11.94\% no tempo de  
computação destes \textit{loops}.

\begin{table}[h]
  \caption{Resultados experimentais.}
  \label{tabela_resultados}
\begin{center}
  \begin{tabular}{|c|c|c|c|}
    \hline
    Algoritmo & Tempo sem otimização & Tempo com otimização & Ganho \%\\
    \hline
    Algoritmo~\ref{read_data_orig} & 470 \textit{ms} & 444 \textit{ms} & 5.85\% \\
    \hline
    Algoritmo~\ref{fix_data_orig} & 485 \textit{ms} & 435 \textit{ms} & 11.26\% \\
    \hline
    Algoritmo~\ref{equalize_orig} & 240 \textit{ms} & 75 \textit{ms} & 320\% \\
    \hline
    Algoritmo~\ref{back_data_orig} & 150 \textit{ms} & 144 \textit{ms} & 4.16\% \\
    \hline
    Algoritmo~\ref{convert_double_orig} & 494 \textit{ms} & 422 \textit{ms} & 17.06\% \\
    \hline
    Algoritmo~\ref{save_file_orig} & 6910 \textit{ms} & 6296 \textit{ms} & 9.75\% \\
    \hline
  \end{tabular}
\end{center}
\end{table}

Considerando o tempo de execução de \textit{wat} na equalização de um
arquivo de áudio \textit{sterio} (2 canais de áudio) de 40 \textit{MB} e
com 4 minutos e 15 segundos de duração, 
o tempo total de execução de \textit{wat} sem otimização é de 19396 \textit{ms} 
e o tempos de execução otimizado é de 18062 \textit{ms}, ou seja, uma melhora de 
7.5\%. 
O percentual de desempenho é menor devido ao fato de que o cálculo da
transformada de Fourier \cite{fft} utilizado em \textit{wat} não pode ser otimizado, uma
vez que, foi utilizado um código já otimizado para este cálculo.


\section{Conclusão e Trabalhos Futuros}

O objetivo desse trabalho é aplicar técnicas de transformação de \textit{loops} 
em uma aplicação para melhora de seu desempenho. 
A aplicação utilizada foi o programa de equalização de áudio \textit{wat}. 
Embora existam muitas técnicas, a obtenção de benefícios em usá-las é uma 
característica empírica. 
Os resultados obtidos com as otimizações mostraram que o uso de técnicas de
transformação de \textit{loops} podem significar um grande impacto no desempenho
do programa.
Casos onde as técnicas não obtiveram resultados significativos, acabou por ficar
evidente que o custo computacional estava em chamadas de função que não podem
ser melhoradas com transformação de \textit{loop}.

Dessa forma, pode-se elencar alguns trabalhos futuros a fim de se chegar em 
resultados ainda mais expressivos do que os aqui obtidos. 

Entre os trabalhos futuros é desejável que se aborde toda a questão de 
transformação de \textit{loops} voltados a uma arquitetura específica. 
Esse tipo de investigação pode ser motivada por possíveis comportamentos 
diferenciados de um mesmo algoritmo conforme a plataforma de 
\textit{hardware} tal como memória \textit{cache} e \textit{intruction cache}.

Técnicas de otimização de código a serem aplicados naqueles que estão no
corpo do \textit{loop} pode ser primordial para que o custo compucional do
\textit{loop} seja reduzido, tal como utilizar funções \textit{inline} e
substituição de variáveis, evitando gasto com acesso a memória através de
ponteiros.

A utilização das ténicas de transformações de \textit{loops} voltadas à
utilização de \textit{threads} em processadores \textit{multi-cores}.




%===============================================================================
\section {Espaço para assinaturas}


\vspace{1cm}

\hfill Londrina, \textit{14 de Novembro de 2013}.


\vspace{1cm}
\noindent
------------------------------------------ \hfill ------------------------------------------

\hspace{1cm} Aluno \hspace{8cm} Orientador

\bibliographystyle{plain} % especifica o estilo como as referências são formatadas
\bibliography{estagio} % especifica o arquivo que contém todas as referências catalogadas.

\end{document}
